\chapter{Introduction}
\section{Initial Situation}
The NoBeard environment is developed to support the courses of Theoretical Informatics conducted for third grade students of the Department of Informatics at the HTBLA Leonding. The project was introduced by Prof. DI Peter Bauer with the purpose that students get the possibility to gain some experience in the field of system programming. 

The NoBeard tools consists of a stack based virtual machine, an assembler and a basic compiler with the corresponding language. The components here are developed in the simplest way  to facilitate the operation of the user as much as possible.

\section{Goals}
The general goal of this thesis was to extend the already existing NoBeard project by a graphical user interface to make the use of the virtual machine more simple. It was the highest priority to meet all the usability requirements. Amongst other things, it includes a better and easier use of the virtual machine compared to command line interfaces. Through the advantages of a GUI users are able to get a better overview of an assembler program execution and its data memory state.

The other major goal of this continuation project was the introducing of a debugger into this system. This tool need to cover nearly all basic debugger functionalities for assembler programs such as toggling breakpoints or stepping through assembler instruction. Of course, the concept should support as well CLI as GUI devices.

\section{Structure of the Thesis}
\subsection{Formal Languages}
This chapter is intended to provide a rough overview of formal languages, their structure (grammars), and usefulness. It gives a short description about the syntax of a language.
