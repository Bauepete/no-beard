\section*{Declaration of Academic Honesty}
Hereby, I declare that I have composed the presented paper independently on my own and without any other resources than the ones indicated. All thoughts taken directly or indirectly from external sources are properly denoted as such.

This paper has neither been previously submitted to another authority nor has it been published yet. \\[1em]
Leonding, \duedateen \\[5em]
\ifthenelse{\isundefined{\firstauthor}}{}{\firstauthor}
\ifthenelse{\isundefined{\secondauthor}}{}{\kern-1ex, \secondauthor}
\ifthenelse{\isundefined{\thirdauthor}}{}{\kern-1ex, \thirdauthor} \\[5em]

\begin{otherlanguage}{german}
\section*{Eidesstattliche Erklärung}
Hiermit erkläre ich an Eides statt, dass ich die vorgelegte Diplomarbeit selbstständig und ohne Benutzung anderer als der angegebenen Hilfsmittel angefertigt habe. Gedanken, die aus fremden Quellen direkt oder indirekt übernommen wurden, sind als solche gekennzeichnet.

Die Arbeit wurde bisher in gleicher oder ähnlicher Weise keiner anderen Prüfungsbehörde vorgelegt und auch noch nicht veröffentlicht. \\[1em]
Leonding, am \duedatede \\[5em]
\ifthenelse{\isundefined{\firstauthor}}{}{\firstauthor}
\ifthenelse{\isundefined{\secondauthor}}{}{\kern-1ex, \secondauthor}
\ifthenelse{\isundefined{\thirdauthor}}{}{\kern-1ex, \thirdauthor} \\[5em]
\end{otherlanguage}

\begin{abstract}
The target of this diploma thesis is to extend an available system programming environment which is called the NoBeard project. The project is developed to enable students to gain some experiences in the field of system programming by coding on assembler level with basic instructions. In addition, the reader of this thesis gets a basic knowledge about formal languages and their uses.

Initially, the project contained three main components:
\begin{itemize}
\item {\em The NoBeard Machine:} A stack based virtual machine with a small set of instructions. The main task of the machine is to execute NoBeard object files.
\item {\em The NoBeard Assembler:} Provided to write programs for the NoBeard Machine.
\item {\em The NoBeard Compiler:} Includes a dedicated programming language and the corresponding compiler to simplify the writing of codes for the NoBeard Machine.
\end{itemize}

The purpose of this thesis was to develop a proper concept of a graphical user interface for the virtual machine.
The concept has to focus the didactic aims to enable users to explore the execution cycle of an assembler instruction, the execution of programs on assembler level, the monitoring of stack frames, the expression stack etc\ldots 

Furthermore, the system is extended with some debugger functions such as setting break points, stepping on assembler instruction level etc\ldots

The initial version of the NoBeard project was developed in Java, therefore the graphical user interface had to be implemented with the Java FX framework.
Primary, it was conceived with the NetBeans IDE on the ant build system, however the entire project was migrated to Maven and further developed using the IntelliJ IDEA environment.
\end{abstract}

\begin{otherlanguage}{german}
\begin{abstract}
Das Ziel dieser Diplomarbeit ist es, eine verfügbare Systemprogrammierumgebung zu erweitern, die als NoBeard-Projekt bezeichnet wird. Das Projekt wurde entwickelt, um den Studierenden zu ermöglichen, einige Erfahrungen im Bereich der Systemprogrammierung zu sammeln, indem sie auf Assembler-Ebene mit grundlegenden Anweisungen kodieren. Darüber hinaus erhält der Leser dieser Arbeit ein grundlegendes Wissen über formale Sprachen und ihre Verwendung.

Ursprünglich enthielt bestand das Projekt aus drei Hauptkomponenten:
\begin{itemize}
\item {\em Die NoBeard Maschine:} Eine stack-basierte virtuelle Maschine mit einer kleinen Menge von Anweisungen. Die Hauptaufgabe der Maschine besteht darin, NoBeard-Objektdateien auszuführen.
\item {\em Der NoBeard Assembler:} Zum Schreiben von Programmen für die NoBeard Maschine.
\item {\em Der NoBeard Compiler:} Enthält eine spezielle Programmiersprache und den entsprechenden Compiler, um das Schreiben von Codes für die NoBeard Maschine zu vereinfachen.
\end{itemize}

Der Zweck dieser Arbeit war es, ein geeignetes Konzept für eine grafische Benutzeroberfläche für die virtuelle Maschine zu entwickeln.
Das Konzept muss die didaktischen Ziele fokussieren, um Benutzern zu ermöglichen, den Ausführungszyklus eines Assemblerbefehl zu erkunden, die Ausführung von Programmen auf Assembler-Ebene, die Überwachung von Stack-Frames etc\ldots

Darüber hinaus wurde das System um einige Debugger-Funktionen, wie das Setzen von Breakpoints, erweitert

Die ursprüngliche Version des NoBeard-Projekts wurde in Java entwickelt, daher musste die grafische Benutzeroberfläche mit dem Java FX-Framework implementiert werden.
Primär wurde es mit der NetBeans-IDE auf dem ant-Build-System konzipiert, jedoch wurde das gesamte Projekt auf Maven migriert und unter Verwendung der IntelliJ IDEA-Umgebung weiterentwickelt.

\end{abstract}
\end{otherlanguage}
