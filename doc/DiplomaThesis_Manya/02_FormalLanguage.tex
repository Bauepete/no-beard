\chapter{Formal languages}
\section{Overview}
A formal language is an abstract language that, unlike concrete languages, often does not focus on communication but on mathematical use. A formal language consists of a certain set of symbol strings (generally strings) that can be assembled from a set of characters or symbols. Formal languages are used in linguistics, logic and theoretical computer science.

Formal languages are suitable for the (mathematically) precise description of the handling of character strings. For example, data formats or entire programming languages can be specified. Together with a formal semantics, the defined character strings have a meaning. With a programming language, a programming instruction (as part of the formal language) can be assigned a unique machine behaviour (as part of the semantics).

Based on formal languages, however, it is also possible to define logic calculi with which one can draw mathematical conclusions. In conjunction with formally defined programming languages, calculi can help to check programs for their correctness.
\bau{cite; generally I would suggest that you add a sentence here, where you give a citation to the book/site from which you took most of the information. Then you do not have to cite this over and over again. Only at the points where you use a different source you add the citation at the point where you use it.}

\section{Strings and Alphabets}
\subsection{Alphabet}
An alphabet (sometimes called a Vocabulary) is a non-empty and finite set of elements. Alphabets are often denoted by upper-case letters $A$ or $V$ and sometimes as upper-case Greek letters like $\Sigma$. Examples of common alphabets are e.g. the machine text alphabet ASCII or the binary bit $0$ or $1$.
\subsection{String}
A string (or word) is a finite sequence of elements, called symbols or letters selected from a particular finite set which is the alphabet.
The length of a string $\omega$ is denoted as $|\omega|$ which gives the number of symbols in the string. The empty string is a special word which has no symbols and is denoted by $\varepsilon$ or $\lambda$. This means a string $\omega$ is called an empty string if $|\omega|=0$.
\subsection{Concatenation}
The basic operation of strings is concatenation, i.e., writing strings as a junction. Let $\omega_1$ and $\omega_2$ be two strings. Then the concatenation of $\omega_1$ and $\omega_2$ makes a new string $\alpha$ containing all the symbols of $\omega_1$ in order followed by all symbols of $\omega_2$ in order. This is usually written as $\alpha=\omega_1\omega_2$.
Concatenations in the alphabet $\{x, y, z\}$ can be:
\begin{align*}
\omega_1&=xxzyyx	& \omega_2&=zxxz	& \omega_1\omega_2&=xxzyyxzxxz\\
\omega_1&=xxzyyx	& \omega_2&=\varepsilon	& \omega_1\omega_2&=xxzyyx\\
\omega_1&=\varepsilon	& \omega_2&=zxxz	& \omega_1\omega_2&=zxxz
\end{align*}

\baust{C}{The c}oncatenation \baust{is associative, that means}{on an alphabet has two important properties}:

\begin{itemize}
	\item Assiciativity: $\omega_1(\omega_2\omega_3)=(\omega_1\omega_2)\omega_3$
	\item An identity element $\varepsilon$: i., e., $\omega\varepsilon = \varepsilon\omega = \omega$
\end{itemize}

In some cases \baust{strings}{string notation} can be simplified by the following rules: 

$a^n$ \qquad For strings containing $n$ same symbols, e.g., $a^3=aaa$

$\varepsilon$\qquad\quad The empty string containing null symbols

$a^+$ \qquad For strings with set of $(a^n:n\geq1)$, e.g., $a^+=\{a, aa, aaa, aaaa, \ldots\}$

$a^*$ \qquad For strings with set of $(a^n:n\geq0)$, e.g., $a^*=\{\varepsilon, a, aa, aaa, aaaa, \ldots\}$
 
\section{Languages and Grammars}
\bau{Here I would start with the Kleeny Star. This helps you to define the term Language pretty formally (see my slides)}
Formally, a language $L$ is defined as a set of strings over some finite alphabet. Specific examples of languages are finite languages with a finite number of words.

For languages we use the usual set-theoretic notations like:
\begin{itemize}
\item $\subseteq$ (inclusion), 
\item $\supseteq$ (proper inclusion), 
\item $\cup$ (union), 
\item $\cap$ (intersection), 
\end{itemize}
\bau{Examples are missing here}

\baust{Belonging of a}{If a} string $\omega$ \baust{in the}{belongs to a} language $L$ \baust{}{it} is denoted by $\omega \in L$, as usual. There are also ``negated'' relations such as $\not\subseteq$, $\not\subset$ and $\not\in$.

\bau{I would move this part after the explanation of production rules and grammars}
The grammatical structure of a statement, a program or generally of a string is a tree, the so-called syntax tree. There are two types of symbols:
\begin{itemize}
\item \textbf{Terminal symbols: }these are the symbols of the sentence itself, i.e. the symbols that can not be further decomposed. In other words they are the ``leaves'' of the tree and are donated with lower-case letter. 
\item \textbf{Non-Terminal symbols: }these are all other symbols with upper-case letters.
\end{itemize}
With $V=V_N \cup V_T$ we denote the union of the alphabets of terminal and non-terminal symbols.
Each tree contains an excellent non-terminal symbol, the start symbol or the root from which the whole tree originates. The allowed structures of syntax trees and thus sentences of a formal language are described by ``grammars''.

A grammar $G(S)$ is a finite, non-empty set of production rules. The rules describe how to build strings from the alphabet of the language that are valid according to the syntax of the language. A grammar does not describe the meaning of the strings or what can be done with them in any context, just their form.

To create a string in the language, we start with a string that consists only of a start symbol. This start symbol $S$ appears on at least one left side of the production rules. All symbols on the left and right sides are the Vocabulary.
\subsection{Production Rule}
A grammar rule has the form:
$$X \rightarrow \alpha \qquad \quad \textrm{With} \quad X \in V_N \quad \textrm{and} \quad \alpha \in V^*$$
It is read ``$X$ is defined as $\alpha$'' or ``$X$ can be replaced by $\alpha$'' and means that in a string containing $X$, $X$ can be replaced by $\alpha$. Several rules with the same left side such as:

$X \rightarrow \alpha_1$

$X \rightarrow \alpha_2$

$X \rightarrow \alpha_3$

are usually grouped together by separating the alternatives with the character ``$\mid$'':
$$X \rightarrow \alpha_1 \mid \alpha_2 \mid \alpha_3$$
The rules describe substitutions of non-terminal symbols by strings. In the formal language theory these substitutions are called production rules. A grammar is defined by production rules that indicate which symbols can replace which other symbols. These rules can be used to generate strings or parse them.

Parsing is the process of recognizing an utterance (a string in natural languages) by breaking it up into a series of symbols and analysing them with the grammar of the language. Most languages have structured the meanings of their utterances according to their syntax.

From the starting symbol $S$, each symbol will be replaced repeatedly by strings, according to the rules of the grammar, until a string is created which contains only terminal symbols.

\subsection{Derivation}
Let a string $\alpha$ generating directly a string $\beta$, written $\alpha \Rightarrow \beta$. Given a formal grammar $G$ with a rule $A \rightarrow \varphi$. If there are the strings $\omega_1$ and $\omega_2$ with the definition $\alpha = \omega_1A\omega_2$, then we can obviously generate a new string $\beta = \omega_1\varphi\omega_2$.
 
A sequence of direct generations is described with the symbols $\Rightarrow^+$ and $\Rightarrow^*$. When a string $\alpha$ produces another string $\beta$ with the form $\alpha \Rightarrow^+ \beta$ then it is about a multiple string generation. A sequence of applications of the rules of a grammar that produces a finished string of terminals is mostly called as derivation. 
$$\alpha = \omega_0 \Rightarrow \omega_1 \Rightarrow \omega_2 \ldots \Rightarrow \omega_n = \beta \qquad \quad \textrm{with} \quad n \geq 1$$ 
In case, that $\alpha \Rightarrow^+ \beta$ or $\alpha = \beta$ we write $\alpha \Rightarrow^* \beta$. This means $\alpha$ generates or is equal to $\beta$.

\subsection{Kleene Star}
The Kleene star set $\Sigma^*$ of an alphabet $\Sigma$ is a language that contains all the words of the alphabet. It can be defined by means of structural induction. 
Firstly, in the beginning of the induction the empty word $\varepsilon$ is defined. Then in the induction step it is defined that each string is being one element in the Kleene star set.
To define the sets $\Sigma_i$ recursively, we use the following components:
\begin{itemize}
\item \textbf{Basis Clause: }$\Sigma_0 = \{\varepsilon\}$ 
\item \textbf{Inductive Clause: }$\Sigma_{i+1} = \{\omega v \mid \omega\in\Sigma_{i-1}\wedge v\in\Sigma\}$
\end{itemize}
The Kleene star is defined then by $\Sigma^* = \bigcup\limits_{i\in\mathbb{N_0}} \Sigma_{i}$
In other words, the Kleene Star of an alphabet is the set of all strings that can be built out of an alphabet.

The Kleene star set $L^*$ of a language $L$ is the union of all its potential languages (repeated concatenation of languages):
$$L^* = \bigcup\limits_{i\in\mathbb{N_0}} \L^{i}$$
Where $L^0 = \{\varepsilon\}$ and $\L^{n+1} = L^nL$. 
\subsection{Syntax of formal grammars}
A formal grammar is represented by the 4-tuple $G=(N,\Sigma,P,S)$ wherein:
\begin{itemize}
\item $N$: a finite set of non-terminal symbols that is separated from the $G$ formed strings.
\item $\Sigma$: a subset of $V$, also called alphabet whose elements are called terminal symbols 
\item $P$: a finite set of production rules where each rule has the form: $(\Sigma \cup N)^{*}N(\Sigma \cup N)^{*}\rightarrow (\Sigma \cup N)^{*}$
\item $S \in N$: the start symbol
\end{itemize}
The set $N = V \setminus T$ is the set of non-terminal symbols (also called meta-symbols), in particular the start symbol belongs to it. The word on the left side of the rule pairs may not consist exclusively of terminal characters, which can also be expressed by a concatenation: $(V^* \setminus T^*) = V^*NV^*$.
\subsection{Example}
Given the grammar $G = (\{S,A,B\},\{a,b,c\},P,S)$ with the non-terminal symbols $S,A,B,$ the terminal symbols $a,b,c$ and the production rule set $P$:

$S \rightarrow Ac$

$A \rightarrow aB \mid BBb$

$B \rightarrow b \mid ab$

Now the following types of strings can be generated by derivation:

\begin{tabular}{llll}
$S \Rightarrow Ac$ & $\Rightarrow aBc$ & $\Rightarrow abc$ & \\
 & & $\Rightarrow aabc$ & \\
 & $\Rightarrow BBbc$ & $\Rightarrow bBbc$ & $\Rightarrow bbbc$ \\
 & & & $\Rightarrow babbc$ \\
 & & $\Rightarrow abBbc$ & $\Rightarrow abbbc$ \\
 & & & $\Rightarrow ababbc$ \\
\end{tabular}

Therefore it is $L(G) = \{``abc", ``aabc", ``bbbc", ``babbc", ``abbbc", ``ababbc"\}$

If $G(S)$ is a formal grammar with a start symbol $S$ then the set $$L(G(S)) = \{\alpha : S \Rightarrow^* \alpha \wedge \alpha \in V^{*}_{T}\}$$
is called the language of $G(S)$.

\bau{If you finish this chapter with different forms of syntax notations (Classical, syntax diagrams, EBNF) then this chapter is pretty ok.}